\chapter{討論與結論}\label{se_8}
\section{討論}\label{se_8} 
下面探討影響私有鏈的共識效率,可能因素。
\begin{itemize}%项目符号开始

\item 區塊大小與網路壅塞問題 :

在TwoStepBFT系統裡,任何訊息的接收都得靠節點與節點間連線傳輸。因此,區塊大小越大,Proposer在廣播區塊時,越有機會發生網路壅塞問題,而拖慢共識效率。


\item 區塊大小選擇與交易速率:

在實驗裡我們透過預先交易,讓礦場內提前充滿交易。且在共識進行時隨時補充新的交易到交易礦池內。在我們的實驗裡 交易產生速率永遠大於共識速率,因此每個共識區塊都填滿了應填滿的交易數量。在選擇區塊大小時,應該考慮應用場景的交易產生速率。

\item 初始Timeout時間與節點數關係:

節點數增加時需廣播的節點數也相對增加。(不論是廣播提議或廣播投票)如果初始的Timeout設定過低,且節點數過多。會頻繁的Timeout
更換新回合。因而拖慢共識效率。


\item 錯誤節點是否連續出現:

在我們的演算法裡,如果錯誤節點擔任Proposer腳色,該回合需等到Timeout並更換新的節點擔任Proposer。Timeout時間也會因此加倍。若錯誤節點連續出現,節點會比錯誤節點非連續出現,需要更多時間等待Timeout。

\item 網路訊息量:

在投票階段訊息傳輸量是$n(n^2)$,如果系統內節點數過多時,會讓訊息傳輸量暴增,影響共識效率。 要降低網路傳輸量,可以將票據交給一個節點再進行傳輸。為了證明節點真的有收到相對應的票數,節點需要附上證明(各節點對票據的簽章)。

\section{結論}\label{se_8}
TwoStepBFT是一種新的兩步驟拜占庭共識演算法,可以保證部分同步假設下的安全性和活躍性。該演算法在任何情況下都只需要兩個步驟,不須額外複雜的方法來達成共識。通過我們的實驗結果,我們知道在AWS 上的測試TwoStepBFT能夠在100個共識節點,擁有數百的吞吐量。並請透過將節點分散搭建在世界各地,更加符合真實應用場景,且吞吐量依舊能維持約300 txs / sec。

透過學習其他共識演算法,TwoStepBFT未來也能繼續改進,改進該演算法 例如.(1)Hotstuff透過聚集簽名,來降低系統傳輸複雜度。(2)由於BFT類的演算法透過輪流廣播訊息,來達到共識,如果共識期間存在錯誤節點,將會拖慢共識效率,如果能將出錯的節點排出共識系統,將能增加安全性及共識效率。

\end{itemize}