\chapter{討論與結論}\label{se_8}
\section{討論}\label{se_8} 
下面我們探討影響共識效率可能因素如:

\begin{itemize}%项目符号开始

\item 節點數影響共識效率 :
節點數量上升會導致傳輸時間拉長,所以完成每個步驟時間都與節點數有關。在演算法設計裡,初始Timeout長度是固定的,如果初始的Timeout設定過低會頻繁的Timeout更換新回合。進而拖慢整體共識效率。

\item 區塊大小影響共識效率:
從實驗裡發現隨著區塊大小變大,共識吞吐量也會隨之增加,但增長幅度有限會趨於平緩;然而共識延遲雖也會隨之增加,但增長幅度卻急遽上升。
因此,我們推斷隨著區塊大小變大,吞吐量將隨之增加,但會趨於平緩甚至下降。然而延遲卻會不斷上升且越來越急遽。因此,我們無法透過無限制增加區塊大小,來增加我們共識效率。


\end{itemize}

\section{結論}\label{se_8}
TwoStepBFT是一種新的兩步驟拜占庭共識演算法,可以保證部分同步假設下的安全性和活躍性。該演算法在任何情況下都只需要兩個步驟,不須額外複雜的方法來達成共識。通過實驗結果,我們知道共識吞吐量與延遲都與節點數息息相關。在AWS 上測試TwoStepBFT效率皆能達到數百TPS。並且透過將節點分散搭建在世界各地,讓實驗更加符合真實應用場景,且吞吐量依舊能維持約300 TPS。

透過學習其他共識演算法,TwoStepBFT未來也能持續改進,例如:(1)Hotstuff透過聚集簽名,來降低系統傳輸複雜度。(2)由於BFT類的演算法透過輪流廣播訊息來達成共識,如果共識期間存在錯誤節點將會拖慢整體共識效率,因此如果能將錯誤節點排出共識系統,將能增加安全性並且提高共識效率。
