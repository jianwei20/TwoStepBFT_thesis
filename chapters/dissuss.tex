\chapter{討論與結論}\label{se_8}
\section{討論}\label{se_8} 
在同區域的實驗裡,結果顯示在區塊大小為1000以上(Txs/Block)時,實驗結果穩定;但在區塊大小翻倍情況下,吞吐量並無大幅上升,但4000與8000的延遲卻大幅上升,故較佳的區塊大小選擇可能為1000(Txs/Block)
在不同區域的實驗裡,吞吐量幾乎皆小於同區域的實驗,延遲也較同區域的實驗高。但即使在不同區域裡,100個節點的共識吞吐量也維持約有300左右。

\section{結論}\label{se_8}
FaS-BFT是一種新的兩步驟拜占庭共識演算法,可以保證部分同步假設下的安全性和活躍性。該演算法在任何情況下都只需要兩個步驟,不須額外複雜的方法來達成共識。通過我們的實驗結果,我們知道在AWS 上的測試FaS-BFT能夠在100個共識節點,擁有數百的吞吐量。並請透過將節點分散搭建在世界各地,更加符合真實應用場景,且吞吐量依舊能維持約300 txs / sec。

透過學習其他共識演算法,FaS-BFT未來也能繼續改進,改進該演算法 例如.(1)Hotstuff透過聚集簽名,來降低系統傳輸複雜度。(2)由於BFT類的演算法透過輪流廣播訊息,來達到共識,如果共識期間存在錯誤節點,將會拖慢共識效率,如果能將出錯的節點排出共識系統,將能增加安全性及共識效率。
