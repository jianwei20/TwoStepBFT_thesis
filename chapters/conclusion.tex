\chapter{相關研究}\label{se_7}
{\bfseries POW:}
比特幣是目前最著名的區塊鏈應用,使用pow作為比特幣的共識系統。在該系統裡節點需要花費時間與電腦運算資源,並嘗試解出一組數學公式的答案,用來獲得廣播區塊的權力。該答案也稱之為nonce。將nonce值附於區塊內,其他節點就能透過簡單數學公式驗證該答案是否有效。除非能控制超過51\%以上的節點,才能進行攻擊。使得pow成為非常安全共識演算法。在私有鏈上因為參與共識的節點較少,因此系統內計算能力總和是非常小的。攻擊者可以輕易地以較高的計算能力,就能推翻過去的共識結果。因此私有鏈不傾向選擇使用POW。
著名的以太坊也使用POW作為共識機制,以太坊除了能夠儲存交易外,還能執行程式碼,運行智能合約,幫助我們執行任務。


{\bfseries BFT(三步驟):}
BFT是拜占庭容錯(Byzantine Fault-Tolerant)的縮寫。在拜占庭容錯的區塊鏈上,即使系統中部分節點當機或存在惡意節點情況下,區塊鏈依舊具備安全性(Safety)與活性(Liveness)等性質。一般來說,一個回合需要執行三個步驟。第一個步驟為播其提議。第二個步驟透過交互投票,檢查該提議者是否只有廣播一份提議。在第三個步驟中,節點便可以開始投票,若一個節點蒐集到足夠多的同意票,便可以確認共識結果。知名的BFT演算法PBFT\cite{castro1999practical}便是根據上述三個步驟設計。而許多私有鏈區外鏈系統,都是根據上述想法而延伸出來。例如:Tendermint[3] 是一種區塊鏈共識機制主要以 Go 語言撰寫,與 PBFT 相似,每個回合都是一次廣播與兩次投票來產生共識,不同於 PBFT 之處在於Tendermint 引入 Lock 概念,藉此維護系統的 Safety 和 Liveness,在官方文件裡提及理想狀況下在 64 個節點能有約 4000 TPS。MSIG­BFT 是一個三步驟的共識演算法,透過收集 f +1 個數位簽章來確保共識提議唯一性,讓共識演算法保證安全性及唯一性,MSIG­BFT 實做在 go­ethereumg上。LibraBFT\cite{STEVE_HANNA2010} 是 Libra 區塊鏈系統所使用的共識系統,在 LibraBFT 技術白皮書提到選擇 Hotstuff\cite{yin2018hotstuff} 作為 LibraBFT 的基礎。Hotstuff 採用了聚集簽名,讓系統內的訊息傳輸量從 $n(n^2)$ 變成 $n(n)$,在 libra 白皮數理提到在理想狀況下 100 個節點能有約 1000tps

{\bfseries BFT(兩步驟):}過去也有少部分的BFT演算法在一個回合中只需要兩個步驟,例子包含FaB\cite{abraham2018revisiting}、Zyzzyva\cite{kotla2007zyzzyva}、SBFT\cite{martin2006fast}、Hydrachain\cite{Hydrachain}。然而,FaB與Zyzzyva已被指出無法保證活性,換句話說,共識演算法可能永遠無法達成共識。另一方面,Hydrachain也被指出無法保證安全性,也就是說,不同的正常節點可能會有不同的共識節果,這在區塊鏈上及代表分岔。雖然SBFT能夠保證安全性與活性,但是在有拜占庭攻擊時,會改用類似PBFT的三步驟設計。因此,在系統狀況良好時,SBFT每回合只需要執行兩個步驟。但是在系統狀況不良時,SBFT每回合仍需要執行三個步驟。在 100 個節點情況下吞吐量約50 TPS。

