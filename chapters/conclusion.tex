\chapter{相關研究}\label{se_7}
Tendermint\cite{buchman2016tendermint} 是一種區塊鏈共識機制主要以Go語言撰寫,與PBFT 相似,每個回合都是一次廣播與兩次投票來產生共識,不同於PBFT之處在於Tendermint 引入Lock概念,藉此維護系統的safety 和liveness,在官方文件裡提及理想狀況下在64個節點能有約4000tps。

SBFT\cite{gueta2018sbft}是一個兩步驟的共識演算法,一樣至少需要5f+1個節點(f是最大能容忍錯誤節點數量),SBFT在一般情況下只需要兩個步驟即能達成共識,但在有拜占庭攻擊時,需要運行額外的步驟,來達成共識,在100個節點情況下擁有50tps

MSIG-BFT是一個三步驟的共識演算法,透過收集$f$+1個數位簽章來確保共識提議唯一性,讓共識演算法保證安全性及唯一性,MSIG-BFT時坐在go-ethereum上,在小節點數時(4個節點)約能有1000TPS。

LibraBFT\cite{STEVE_HANNA2010}是Libra區塊鏈系統所使用的共識系統,在LibraBFT技術白皮書提到選擇 Hotstuff\cite{yin2018hotstuff} 作為 LibraBFT 的基礎。Hotstuff採用了聚集簽名,讓系統內的訊息傳輸量從$n$($n^2$) 變成 $n$($n$),在libra白皮數理提到在理想狀況下100個節點能有約1000tps
