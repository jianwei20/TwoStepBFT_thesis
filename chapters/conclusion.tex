\chapter{相關研究}\label{se_7}
\section{公有鏈上共識演算法}\label{se_7}
\subsection{Proof-of-Work}\label{se_7} 
比特幣是目前最著名的區塊鏈應用,使用PoW(Proof-of-Work)作為比特幣的共識系統。在該系統裡節點需要花費時間與電腦運算資源,嘗試解出一組數學公式的答案,用來獲得廣播區塊的權力。該答案也稱之為Nonce。將Nonce值附於區塊內,其他節點就能透過簡單數學公式,驗證該答案是否有效。除非能控制超過51\%以上的系統運算能力,才能進行攻擊。在私有鏈上因為參與共識的節點較少,因此系統內計算能力總和非常小。攻擊者可以輕易地以較高的計算能力,推翻過去的共識結果,因此私有鏈不傾向選擇使用PoW。目前比特幣約7TPS \cite{BitcoinThroughput}
著名的以太坊 \cite{Ethereum}、萊特幣 \cite{Litecoin}也都使用PoW作為共識機制。以太坊除了能夠儲存交易外,還能執行程式碼,運行智能合約。
\subsection{Proof-of-Stack}\label{se_7}
PoS(Proof-of-Stack)是把持有資產數量作為參考。資產數量越多者,越有機會擔任下個區塊的廣播者。PoS認為持有資產量高的人會越傾向維護貨幣價值,因此發動攻擊的可能性越低。不過PoS缺點也非常明顯,PoS很可能造成貨幣不流通。如果貨幣不流通,該貨幣也失去其價值。PPCoin \cite{vasin2014blackcoin}是目前少數運行PoS的區塊鏈。
\subsection{混合型共識演算法}\label{se_7}
混合型共識演算法概念是從公有鏈節點中,隨機挑選一群人運行拜占庭容錯共識演算法。這類的方法需要具有非常公平的抽籤演算法。Algorand \cite{gilad2017algorand}作為一個混合型共識演算法,最重要的一個機制便是引入了VRF(Verifiable Random Function),中文稱作可驗證隨機函數。透過該函數,區塊鏈上節點能夠自行驗證是否成為該回合的提議者,並且能夠提出證明供其他節點進行驗證。Algorand的處理能力超過1000TPS且延遲低於五秒。相似的混合型演算法還包含Bitcoin-NG \cite{eyal2016bitcoin}、Dexon \cite{dexon}


\section{私有鏈上共識演算法}\label{se_7}
\subsection{Proof-of-Authority}\label{se_7}
PoS(Proof-of-Authority)是由一群授權的節點來負責驗證區塊與廣播區塊。不同於PoW驗證節點不需強大的運算能力,也不需要像PoS得擁有很多資產才能廣播區塊。但此節點必須是大家公認的已知節點,且通過一定程度身分驗證。一旦節點勾結其他節點進行攻擊,那鏈上的其他管理者可以移除或替換這些惡意節點。目前以太坊在原始碼中也提供PoA共識演算法名為Clique \cite{clique}共識演算法。

\subsection{三步驟拜占庭共識演算法}\label{se_7}

BFT是拜占庭容錯(Byzantine-Fault-Tolerant)的縮寫。在拜占庭容錯的私有區塊鏈上,即使系統中部分節點當機或存在惡意節點情況下,區塊鏈依舊具備安全性(Safety)與活性(Liveness)等性質。一般來說,一個回合需要執行三個步驟。第一個步驟為播其提議;第二個步驟透過交互投票,檢查該提議者是否只有廣播一份提議;第三個步驟節點便開始投票,若一個節點蒐集到足夠多的同意票,便可以確認共識結果。知名的BFT演算法PBFT \cite{castro1999practical}便是根據上述三個步驟設計。而許多私有鏈區外鏈系統,都是根據上述想法而延伸出來。例如:Tendermint[3] 是一種區塊鏈共識機制主要以 Go 語言撰寫,與 PBFT 相似,每個回合都是一次廣播與兩次投票來產生共識,不同於 PBFT 之處在於Tendermint 引入 Lock 概念,藉此維護系統的 Safety 和 Liveness。在官方文件裡提及理想狀況下 64 個節點能有約 4000 TPS。MSIG­-BFT 是一個三步驟的共識演算法,與一般BFT演算法不同的是透過收集 f +1 個數位簽章來確保共識提議唯一性,讓共識演算法保證安全性及唯一性。目前MSIG­-BFT 實做在 Go­-ethereum上。LibraBFT \cite{STEVE_HANNA2010}是 Libra 區塊鏈系統所使用的共識系統,在 LibraBFT 技術白皮書提到,選擇 Hotstuff \cite{yin2018hotstuff} 作為 LibraBFT 的演算法基礎。Hotstuff 採用了聚集簽名,讓系統內的訊息傳輸量從 $O(n^2)$ 變成 $O(n)$,在 libra 白皮書裡提到,理想狀況下 100 個節點仍有約 1000 TPS。

\subsection{兩步驟拜占庭共識演算法}\label{se_7}
過去也有少部分的BFT演算法在一個回合中只需要兩個步驟,例子包含FaB \cite{abraham2018revisiting}、Zyzzyva \cite{kotla2007zyzzyva}、SBFT \cite{martin2006fast}、Hydrachain \cite{Hydrachain}。然而,FaB與Zyzzyva已被指出無法保證活性,換句話說,共識演算法可能永遠無法達成共識。另一方面,Hydrachain也被指出無法保證安全性,也就是說,不同的正常節點可能會有不同的共識節果,這在區塊鏈上及代表分岔。雖然SBFT能夠保證安全性與活性,但是在有拜占庭攻擊時,會改用類似PBFT的三步驟設計。因此,在系統狀況良好時,SBFT每回合只需要執行兩個步驟。但是在系統狀況不良時,SBFT每回合仍需要執行三個步驟。在 100 個節點情況下吞吐量約50 TPS。


