\chapter{系統模型}\label{se_2}
我們在2.1節描述網路模型與系統假設並在2.2節描述本論文共識假設與論文常見符號定義
\section{網路模型與系統假設}\label{se_2} 

早在1985年Fische等人,提出了FLP不可能證明\cite{fischer1982impossibility},該論文證明在一個完全異步的網路系統中,假使存在節點失效(即便只有一個節點失效),不存在一個可以達成共識的演算法。所以我們將系統假設為部分同步的網路延(Partially synchronous)模型。更明確地說,若$D(t)$代表一個在時間$t$傳遞訊息所會遭遇的延遲時間,則$D(t)$成長的速度不可能永遠大於$t$的成長速度。部分同步的網路延遲模型具有以下的性質:若是每個回合的長度都是上個回合長度的兩倍,則一定存在一個回合$i$,在回合$i$中前半段傳遞的訊息都能在回合$i$的後半段收到。因此,在我們設計的BFT演算法中,每個回合的長度會是前一個回合長度的兩倍。

\section{共識假設與定義}\label{se_2} 

我們假設$n$為共識問題中的參與共識的總節點數量(包含故障節點的數量),$f$則為錯誤節點上限。Martin與Alvisi在Fast Byzantine Consensus該論文\cite{martin2006fast}證明了,如果要設計一個拜占庭容錯共識演算法,且每個回合只能執行兩個步驟,為了確保系統共識結果一致,$n$必須至少大於5$f$。因此,我們假設$n$ = 5$f$ + 1。

%
下列我們定義一些共識演算法中的基本元素 
\begin{itemize}%项目符号开始
\item  區塊:以太坊的基本區塊包含 Header 與交易以下會以 $b$ 表示。
\item  節點:參與共識演算法的個別節點以$u$。
\item  高度:共識的區塊高度以 $h$ 表示,區塊鏈會從 $h=0$開始進行共識。
\item  回合:每次共識回合則用 $r$ 表示,每回合會從 $r=0$開始進行共識。
\item  提議:在每個回合開始時我們會從所有參與共識的節點中,選出一位Proposer進行廣播,由它提出當回合共識的區塊 $b$。我們稱Proposer其所廣播的提議為Proposal,以$p$ 表示。我們以$p(h,r,u,b)$表示此Proposal是在第 $h$ 的高度、第 $r$ 的回合、來自於Proposer $u$ 且包含一個區塊 $b$。
\item  Vote:Vote 為一種訊息型態,在共識期間節點之間會互相廣播Vote。我們以 $v(h,r,p,u,b)$表示一張Vote,代表此 Vote 在第$h$的高度、在第$r$的回合、投給提議$p$且來自於 $u$ 節點。在演算法系統裡,Vote 會紀錄區塊 $b$ 的Hash 值而非整個區塊結構。

\item  Lockset:節點會將Votes儲存於Lockset以進行將來的共識。我們以$l(h,r,u)$ 表示一個 Lockset 在第 $h$的高度、在第 $r$ 的回,來自於 $u$ 節點。且所有在此 $l(h,r,u)$裡,每一張收到的$v(h,r,p,u,b)$中,$h$與$r$都是必須相同,但這裡Vote所投的區塊卻不一定需要相同。一個Lockset 需要包含大於4/5 $n$ 的Votes才能成為一個合法的 Lockset ,且Vote 必須來自於不同節點。
\item Timeout: Timeout定義了段一個時間長度,在演算法裡Timeout分成了接收Proposal的時間閘$TOvote$與接收Vote的時間閘$TOcommit$,在我們的演算法裡$TOvote$ = $TOcommit$。 

\end{itemize}