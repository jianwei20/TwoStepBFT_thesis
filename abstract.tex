\begin{abstractzh}
區塊鏈技術至今已發展十多年歷程。區塊鏈應用也從一開始數位貨幣衍伸生出多樣化的應用與服務。區塊鏈是一種分散式帳本技術,帳本內容是由多個網路節點共同維護,而非受到單一節點所控制。為了確保安全性,系統內所有節點需要擁有相同帳本資料;換句話說,節點間須對帳本內容達成共識,而達成共識的方法稱共識演算法。為因應不同的應用情境,區塊鏈分為公有鏈與私有鏈。公有鏈對所有人皆開放,而私有鏈只開放特定人群或企業加入。本篇論文針對私有鏈上的共識演算法進行探討及研究。一般而言,私有鏈共識演算法需要三個步驟的訊息交流,才能確保共識結果是正確的。我們設計了一個兩步驟的共識演算法-TwoStep­BFT。此演算法能夠容許拜占庭節點錯誤,且保有安全性及活性。為了架設大規模節點的私有鏈,我們整合了多樣的自動化雲端部署套件,此套件幫助我們在雲端平台上自動產生多個節點。實驗結果顯示我們的方法在一百個節點依舊能有300TPS的共識效率。
\noindent
關鍵字: 區塊鏈、共識演算法、拜占庭容錯。 
\end{abstractzh}

\begin{abstracten}
Blockchains have been developed for more than a decade. The very first application of blockchains is digital currency. Within a decade, blockchains have found applications in various fields. A blockchain can be viewed as a distributed ledger, whose content is maintained by multiple network nodes rather than by a single node. In order to ensure correctness, all nodes in the blockchain need to have the same content of the ledger. In other words, all nodes must reach consensus on the content of the blockchain. Toward this goal, all nodes must execute the same protocol that specifies the rules of adding new blocks to Blockchain. Such a protocol is called a consensus algorithm. There are two types of blockchains, public blockchains and private blockchains. Public blockchains have no access control, and is open to everyone. On the other hand, private blockchains can only be accessed by admitted nodes. This thesis focuses on the design and analysis of consensus algorithms for private blockchains. In general, consensus algorithms for private blockchains involve three steps of information exchange to ensure correctness. In this thesis, we design a two-step Byzantine tolerant consensus algorithm, TwoStepBFT. In order to evaluate the performance of TwoStepBFT on a 100-node blockchain, we integrate a variety of automated tools to deploy blockchains. Experimental results have shown that our consensus algorithm can reach a throughput of 300 TPS on a 100-node Blockchain.
	
\noindent
Keywords: Blockchain, Consensus Algorithm, Byzantine Fault Tolerance, Automation. 
\end{abstracten}
