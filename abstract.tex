\begin{abstractzh}
區塊鏈技術至今已發展十多年歷程,區塊鏈應用也從一開始數位貨幣衍伸出多樣化的應用與服務。因應不同的使用情境,區塊鏈衍生出公有鏈與私有鏈。公有鏈對所有人皆開放,而私有鏈只開放特定人群或企業加入。區塊鏈主要特性為去中心化。意即區塊鏈非由單一網路節點所控制,而是由多個網路節點共同維護。為了使網路上多個節點共同維護區塊鏈,需要共識演算法,讓區塊鏈內容達成一致。本篇論文針對私有鏈上的共識演算法進行探討及研究。一般而言,共識演算法需要三個步驟的訊息交流,才能確保共識結果是正確的。我們設計了一個兩步驟的共識演算法,TowStep­BFT。此演算法能夠容許拜占庭節點錯誤,且保有足夠的安全性及活性。為了架設大規模節點的私有鏈,我們整合了多樣化的自動化雲端部屬套件。幫助在雲端平台上,自動化產生節點並分析實驗結果。實驗結果顯示我們的方法在一百個節點依舊能有300TPS的共識效率。\\

\noindent
關鍵字: 區塊鏈、共識演算法、拜占庭容錯。 
\end{abstractzh}

\begin{abstracten}
Blockchain technology has been developed for more than ten years. Deriving from digital currency, blockchain also expands diversified applications and services. The key property of blockchain is decentralization, meaning that content is not generated by specific nodes or people but created by multiple nodes in the system with equal rights. The core technology of decentralization is the consensus algorithm. This paper discusses about the consensus algorithm on the private chain. Generally speaking, the consensus algorithm usually requires three steps of information exchange to ensure that the consensus result is unique and safe. we designed a two steps consensus algorithm, FaS-BFT, which allows the Byzantine attack and maintains safety. In order to experiment in hundreds of nodes, we integrated a variety of automated tools to deploy private blockchain and analyze experimental results on the cloud platform. Experimental results have shown that our consensus efficiency still has well performance on hundreds of nodes. \\
	
\noindent
Keywords: Blockchain, Consensus Algorithm, Byzantine Fault Tolerance, Automation. 
\end{abstracten}
